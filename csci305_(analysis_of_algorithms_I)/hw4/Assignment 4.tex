

\documentclass[12pt]{article}
\usepackage{color} %used for font color
\usepackage{amssymb} %maths
\usepackage{amsmath} %maths
\usepackage{amsthm}
\usepackage{amscd}
\usepackage{fancyhdr}
\usepackage[all]{xy}
\usepackage[utf8]{inputenc} %useful to type directly accentuated characters
\usepackage{enumitem}
\usepackage{graphicx}
\usepackage{hyperref}
\usepackage{algorithm}
\usepackage{algorithmic}
\usepackage{graphicx}
\graphicspath{{../pdf/}{/Users/username/Desktop}}

\pdfpagewidth 8.5in
\pdfpageheight 11in

\setlength\topmargin{-1in}
\setlength\headheight{1in}
\setlength\headsep{.5in}
\setlength\textheight{8.5in}
\setlength\textwidth{6.5in}
\setlength\oddsidemargin{0in}
\setlength\evensidemargin{0in}

\usepackage{mathtools}
\DeclarePairedDelimiter\floor{\lfloor}{\rfloor}

\usepackage[parfill]{parskip} 

\setlength{\parindent}{0pt}

\usepackage{forest}


\begin{document}

\pagestyle{fancy}
\lhead{Alec Harb
\\ March 12, 2020}
\rhead{CSCI 305}
\chead{\Large Assignment 4}

%\vspace{5mm} %%this just adds a blank line if you need it

\section{Question 1}



\subsection{a.}
\subsection{b.}
\subsection{c.}

\section{Question 2}

\subsection{a.}

The successor of $x$ is the smallest element in the right subtree of $x$ in a BST. Let $r$ be $x$'s successor. If $r$ has no right child, it will have $0$ children because if it had a left child, it would not be the successor of $x$/. If $r$ has a right child, that child will be $r$'s only child for the reason just stated. Therefore, $x$'s successor can have either $0$ or $1$ children.

\subsection{b.}

The predecessor of $x$ is the largest element in the left subtree of $x$. Let $l$ be $x$'s predecessor. Then, if $l$ has no right child, it will have $0$ children because a right child would imply that $l$ is not $x$'s successor. If $l$ has a left child, it will still be the predecessor to $x$ because any child in the left subtree of $l$ will be strictly less than $l$. Therefore, $l$ can have $0$ or $1$ children.

\section{Question 3}

\subsection{a.}

funA:\\
Let $x$ be an arbitrary node in a BST. In every instance of comparing $x$'s children to $x$, each invocation ensures that no left child be greater than $nodeVal(x)$ and no right child can be less than $nodeVal(x)$. This implies that funA is a correct algorithm.

funB:\\
I will give an example to show that the algorithm is incorrect. Suppose we take an arbitrary node $x$ in the BST. Let $l$ be the left child of of $x$ and $l_r$ be the right child of $l$. Given this, the algorithm will not check if $nodeVal(l_r) < nodeVal(x)$ because it is only comparing children nodes to their respective parent nodes. So, if $nodeVal(l) < nodeVal(x)$ but $nodeVal(l_r) > nodeVal(x)$, funB will still return true.

\subsection{b.}

funA: \\
At each check of a node we are using $tree-get-max()$ and $tree-get-min$ functions which have a runtime of $O(h)$.  Additionally, since we are calling both functions for each node, we have a runtime of $2h * n$. Since $h$ is a constant, we drop it to get a complexity of $\theta(hn)$. In the best case, where the BST properties do not hold, funA will have a complexity of $\Omega(h)$ because it will still call both $get-max/min$ functions in order to determine the tree is not a BST. 

funB:\\
For each node in the BST, we are compare it to its parent, left child, and right child which, implies that each node is compared to $3$ times. Since we compare every node in the tree, with a size of $n$, we have a runtime of $3n$. Since $3$ is a constant we can drop it, which yields a complexity of $\theta(n)$.

\section{Question 4}

\subsection{a.}

From the book, we have that the expected upper bound for the number of probes of an unsuccessful search is $\frac{1}{1-\alpha}$. Using $\alpha = \frac{6}{7}$ we have $\frac{1}{1- \frac{6}{7}} = 7$ probes. 

\subsection{b.}

Also from the book, we have that the expected upper bound for the number of probes of a successful search is $\frac{1}{\alpha}ln(\frac{1}{1-\alpha})$. Using $\alpha = \frac{6}{7}$ we have 
$\frac{1}{ \frac{6}{7}}ln(\frac{1}{1- \frac{6}{7}}) = \frac{7}{6}ln(7) = 2.27$ probes.

\section{Question 5}

From (4), we know the equations of the expected number of probes. For this problem, we can set the equation for the unsuccessful search equal to $3.5$ times the equation for the successful search.
This gives $\frac{1}{1-f} = \frac{3.5}{f}ln(\frac{1}{1-f})$. We can then graph these two equations and see where they intersect. When graphed, both lines intersect when $x = 0.882$. Thus, a load factor $f$ of 0.882 will ensure that the expected number of probes in an unsuccessful search is $3.5$ times the expected number of probes in a successful search. 

\end{document}
